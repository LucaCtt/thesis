\chapter{Background}\label{ch:background}

A number of studies have been conducted on the use of a \acrfull{dt} in smart homes. One of the prominent topics in the literature is the use of \acrshort{dt}s for monitoring the health status of the inhabitants in smart homes. For instance, \textcite{chenDigitalTwinEmpowered2023} presents a \acrshort{dt} model that integrates the users' electrocardiograph waves and the WiFi signals in the house to provide graphical monitoring, healthcare prediction, and intelligent control. Similarly, \textcite{shoukatSmartHomeEnhanced2023} proposes a framework that combines the information from the smart home sensors with the inhabitants' health records to enhance their well-being and safety. Another example is the work by \textcite{bouchabouSmartHomeDigital2023}, who develop a tool and a methodology for creating \acrshort{dt}s for smart homes, which include a simulator for monitoring the daily activities of the inhabitants and a replicable approach for modeling realistic apartments and scenarios.

Another domain where \acrshort{dt}s can be applied in smart homes is energy management, which involves optimizing the energy consumption and generation of the smart home system. For example, \textcite{fathyDigitalTwinDrivenDecision2021} proposes a data-driven multi-layer \acrshort{dt} of the energy system that aims to mirror the actual energy consumption of the households. The \acrshort{dt} can then be used to reorganize the energy consumption patterns of the residential homes to avoid peak demands, while satisfying the resident needs and reducing their energy costs. However, this work does not take into account the individual preferences and comfort of the inhabitants. This aspect is considered by \textcite{huangMachineLearningbasedDemand2023}, who presents an hour-ahead demand response scheme for smart homes that utilizes \acrshort{ai} methods in the \acrshort{dt} to balance the customer energy consumption and discomfort.

The following sections provide an overview of the main concepts and technologies that are relevant to the work presented in this thesis.

\newpage

\section{Digital Twins}

A \acrshort{dt} is a computer-based model that simulates, mirrors, or ``twins'' the life of a physical entity, such as an object, a process, a human, or a human-related feature \parencite{barricelliMultiModalApproachCreating2022}.

A \acrshort{dt} is more than a passive copy; it is an intelligent and adaptive system that acts as the living counterpart to its physical twin \parencite{grievesDigitalTwinManufacturing2015,kritzingerDigitalTwinManufacturing2018}. It continuously  monitors, analyzes, and optimizes the real-world entity throughout its lifecycle \parencite{negriReviewRolesDigital2017}. This allows for proactive maintenance and prevention of issues like defects and failures, as well as experimentation and optimization through simulations of new scenarios. The twinning process is enabled by the dynamic interplay between the \acrshort{dt}, its physical twin, and the surrounding environment - a continuous loop of interaction, communication, and improvement.

The \acrshort{dt} keeps track of its physical twin and environment through real-time data streams and extensive data storage capabilities. Descriptive data exchanges, facilitated by big data infrastructure, ensure consistent alignment with the physical system. Advanced algorithms, including data fusion, big data analytics, and \acrshort{ai}-based descriptive methods, are used to extract valuable insights from this rich data source. The modular and highly parameterized architecture of the \acrshort{dt} allows for quick reconfiguration, enabling it to evolve along with its physical twin. This synchronization ensures the \acrshort{dt} always reflects the properties and changes happening in the real world.

Leveraging its \acrshort{ai} capabilities, the \acrshort{dt} goes beyond mere emulation by discovering hidden patterns, unknown correlations, and comprehensive system descriptions. This holistic understanding of the physical system, along with the ability to record, control, and monitor its state, enables techniques for failure prediction, solution simulation, and even self-healing mechanisms. This powerful combination enables a predictive maintenance approach, where proactive actions prevent failures and optimize performance through simulated testing and solution selection.

Despite their inherent intelligence, \acrshort{dt}s are not meant to operate in total autonomy. \acrshort{ai}-based applications and \acrshort{dt}s often need significant human input, especially when testing new features and modifications on physical assets, or when the systems are used to provide critical outputs like medical diagnoses and treatment recommendations \parencite{barricelliSurveyDigitalTwin2019}.

\subsection{Historical Notes}

The term \acrshort{dt} was coined by Michael Grieves in 2002, in the context of \acrfull{plm} \parencite{grievesDigitalTwinManufacturing2015,grievesDigitalTwinMitigating2017}. He defined a \acrshort{dt} as a virtual representation of a physical object, connected by a bidirectional flow of data and information. Figure~\ref{fig:grieves_digital_twin} illustrates his \acrshort{dt} model, which consists of three main components:
\begin{enumerate}
    \item A physical space, where the real object exists and operates;
    \item A virtual space, where the digital object is created and simulated;
    \item A connection, which enables data to be transferred from the physical to the virtual space (and its sub-spaces), and information to be sent from the virtual space (and its sub-spaces) to the physical space. This connection is essential for achieving alignment and synchronization between the virtual and physical systems.
\end{enumerate}

\begin{figure}
    \centering
    \includegraphics[width=0.8\textwidth]{images/digital_twins/dt_model_grieves.png}
    \caption[Grieves' digital twin model]{Grieves' digital twin model. From \textcite{barricelliSurveyDigitalTwin2019}, \textit{Research Background} section, CC-BY 4.0 license.}
    \label{fig:grieves_digital_twin}
\end{figure}

Building on Grieves' idea, \textcite{framlingProductAgentsHandling2003} proposed ``an agent-based architecture where each product item has a corresponding virtual counterpart or agent associated with it''. They envisioned that these agents would leverage the ubiquitous connectivity of the Internet to maintain synchronization with their physical counterparts, and also offer services for them \parencite{framlingProductAgentsHandling2003}. Their motivation was that a robust \acrshort{plm} system should have a reliable and up-to-date view of the product's status and information, throughout its entire lifecycle, from design and production, to usage and disposal.

\acrshort{dt}s garnered much interest in the aerospace and defense industries~\parencite{negriReviewRolesDigital2017}. NASA explored \acrshort{dt}s as a way to save costs and resources for its space missions. Their study resulted in a roadmap that validated the value of \acrshort{dt}s in enhancing performance in the aviation domain. They described the DT as ``an integrated multi-physics, multi-scale, probabilistic simulation of a vehicle or system that uses the best available physical models, sensor updates, fleet history, etc., to mirror the life of its flying twin. The digital twin is ultra-realistic and may consider one or more important and interdependent vehicle systems, including propulsion/energy storage, avionics, life support, vehicle structure, thermal management/TPS, etc.''~\parencite{shaftoModelingSimulationInformation2010}. The U.S. Air Force also created the \acrfull{adt}, a computational model of individual aircrafts, which aimed to improve the way U.S. Air Force aircrafts were maintained over their entire lifecycle by generating personalized structural management plans~\parencite{tuegelAirframeDigitalTwin2012,gockelChallengesStructuralLife2012}.

Moreover, \acrshort{dt}s are also a key element of Industry 4.0~\parencite{brettelHowVirtualizationDecentralization2014,hermannDesignPrinciplesIndustrie2016,vachalekDigitalTwinIndustrial2017,negriReviewRolesDigital2017}, where they are applied to monitor and optimize the performance of manufacturing systems, and to facilitate the development of new products and services. The \acrshort{dt} is a catalyst for the digital transformation of the manufacturing industry, and it is a vital component of the smart factory \parencite{mabkhotRequirementsSmartFactory2018}.

\iffalse
    \subsection{Characteristics of Digital Twins}

    Seamless and reliable communication is fundamental to \acrshort{dt} functionality. Both physical and virtual twins require networking devices to facilitate continuous data exchange, achieved either through direct physical connections or indirect cloud-based networks. This facilitates a constant flow of information from the physical twin, which describes the physical twin status and change with time along its lifecycle, along with dynamic environment data describing the surrounding environment status. Furthermore, the \acrshort{dt} actively transmits predictions for maintenance, optimizations, insights, and recommendations for improved function to the physical twin, human specialists, and to other \acrshort{dt}s in its environment. As such, three key communication processes must be designed:

    \begin{enumerate}
        \item Physical-\acrshort{dt}. This direct exchange ensures real-time synchronization between the physical system and its digital representation.
        \item \acrshort{dt}-to-\acrshort{dt}: This enables collaboration and knowledge sharing between multiple digital twins within the interconnected environment.
        \item \acrshort{dt}-Expert: User-friendly interfaces facilitate interaction and operation of the \acrshort{dt} by domain experts, maximizing the value derived from this data-driven technology.
    \end{enumerate}

    The \acrshort{dt} relies on a robust data storage system to house the continuously streamed sensor data reflecting the real-time status and changes of the physical twin.
    It also memorizes historical static data, which reflect the physical twin memory and record historical information provided by human expertise or by past actions, and descriptive static data, that captures essential, unchanging characteristics of the physical twin. To comprehend and formalize this diverse data, the \acrshort{dt} leverages proper \textit{ontologies}, i.e. shared, machine understandable vocabularies for information exchange among dispersed agents (e.g. humans and different machines), ensuring consistent interpretation and efficient utilization~\parencite{negriReviewRolesDigital2017}.

    Given the complexity of captured data, the \acrshort{dt} employs techniques for high-dimensional data (de-)coding and analysis, which are tailored to process and analyze complex, multi-dimensional data structures. It also requires data fusion algorithms, to integrate data from various sources like sensors and historical records, providing a holistic understanding of the physical system.

    The \acrshort{dt} includes various \acrshort{ai} algorithms, whose predictive capabilities are constantly refined. Supervised and/or unsupervised machine learning models to classify and categorize data points based on past observations and known patterns. Their predictive capability is refined as they process the continuously received sensed data from the physical twin and the surrounding environment. Feature selection and extraction methods are used to reduce the data dimensionality while keeping the most informative data, enahncing computational efficiency.

    Finally, the \acrshort{dt} has self-adaptation and self-parametrization capabilities, to adjust its internal parameters to keep pace with the evolving physical twin throughout its lifecycle. It also exploits predictive analytics to predict future statuses, important changes, and to recommend optimal actions and interventions. When making future predictions, it must take into account uncertainties in the data. At any time, it should offer a real-time view of the physical system's condition, and enable simulation and exploration of potential \textit{what-if} scenarios under various conditions, supporting informed decision-making~\parencite{boschertDigitalTwinSimulation2016}.
\fi

\subsection{Design of Digital Twins}

\acrshort{dt}s can have two different lifecycles, depending on whether the objects they represent are already existing or not~\parencite{barricelliSurveyDigitalTwin2019}. The first lifecycle applies to objects that are still in the design stage, and involves creating both the object and its \acrshort{dt} simultaneously. The second lifecycle applies to objects that already exist, but do not have a \acrshort{dt} yet. In this case, the design process aims to equip the objects with connectivity features to enable their \acrshort{dt} creation. Both lifecycles follow the same sequence: a Design phase, a Development phase, an Operational phase, and a Dismissal phase.

\begin{figure}
    \centering
    \includegraphics[width=0.9\textwidth]{images/digital_twins/dt_lifecycle_1.png}
    \caption[Lifecycle of a \acrshort{ct} scanner and its \acrshort{dt}, starting from the development of the \acrshort{dt}]{Lifecycle of a \acrshort{ct} scanner and its \acrshort{dt}, starting from the development of the \acrshort{dt}. From \textcite{barricelliSurveyDigitalTwin2019}, \textit{Design Implications} section, CC-BY 4.0 license.}
    \label{fig:dt_lifecycle_1}
\end{figure}

To illustrate these two lifecycles, a \acrfull{ct} scanner is used as an example. The first lifecycle is shown in Figure~\ref{fig:dt_lifecycle_1}.

The \acrshort{dt} is created before the physical object as a prototype $DT_{object}$, which assists the designers in the design phase of the prototype object. The prototype $DT_{object}$ serves as a virtual model of the real prototype, enabling the designers to simulate, test, modify, and validate their design choices using data from the prototype $DT_{object}$ and other sources. When the prototype $DT_{object}$ is finalized, the design phase shifts to the prototype object, where the prototype $DT_{object}$ may be revised to address any technical issues. In the development phase, the prototype $DT_{object}$ becomes a development $DT_{object}$, which interacts with the production machines to monitor and optimize the assembly/construction of the object, i.e., its physical twin. When the object is built, the development $DT_{object}$ becomes a product $DT_{object}$, and the operational phase begins. The product $DT_{object}$ corresponds to the object and tracks and mirrors the \acrshort{dt} scanner while it is in use. The product $DT_{object}$ also learns and adapts to the object during its operation. When the object is no longer used, the dismissal phase starts, first for the object and then for the $DT_{object}$. The historical data of the product $DT_{object}$ are backed up and shared with other \acrshort{dt}s and domain experts, so that they can use the information to improve the production of future devices.

\begin{figure}
    \centering
    \includegraphics[width=0.9\textwidth]{images/digital_twins/dt_lifecycle_2.png}
    \caption[Lifecycle of a \acrshort{ct} scanner and its \acrshort{dt}, when the \acrshort{ct} scanner already exists]{Lifecycle of a \acrshort{ct} scanner and its \acrshort{dt}, when the \acrshort{ct} scanner already exists. From \textcite{barricelliSurveyDigitalTwin2019}, \textit{Design Implications} section, CC-BY 4.0 license.}
    \label{fig:dt_lifecycle_2}
\end{figure}

The second lifecycle is in Figure~\ref{fig:dt_lifecycle_2}. In this case, the object is already existing and in use, but it does not have a connected DT. The design phase creates a new prototype DT ($DT_{object}$), which is tested, modified and validated. The development phase connects the existing object and the $DT_{object}$, now a development $DT_{object}$. The operational phase is the life of the two twins, the connected object and the product $DT_{object}$, which work together until their disposal in the dismissal phase. The (physical and digital) twins rely on a synergistic and continuous interaction, which enables monitoring, predicting, and optimizing all their functions.

\section{Automations for Smart Devices}

Smart home control devices, such as Google Nest Mini, Amazon Echo Show, and Apple HomePod, are becoming more and more popular and accessible in recent years. They allow users to manage their home's \acrfull{iot} ecosystem, which consists of various smart devices, social networks, applications, recommender systems, and users~\parencite{barricelliDesigningEndUserDevelopment2015}. These smart devices embed electronic components that enable user interaction and control. For example, by applying \acrshort{iot} to the smart home, users can remotely or automatically adjust the lights, televisions, thermostat, shutters, locks, humidity sensors, and other smart objects in their home~\parencite{kortuemSmartObjectsBuilding2010,wuRespectChangeUser2017}.

To meet the diverse and changing needs of users, smart home control devices should provide them with the ability to personalize their smart home. The term \textit{end users} refers to people who are not experts in programming or computer technologies, but who use computer systems for their daily activities, such as work or entertainment. Therefore, \acrfull{eud} techniques are employed to facilitate this personalization. \acrshort{eud} can be described as “the set of methods, techniques, tools, and socio-technical environments that allow end users to act as professionals in those ICT-related domains in which they are not professionals, by creating, modifying, extending and testing digital artifacts without requiring knowledge in traditional software engineering techniques”~\parencite{barricelliEnduserDevelopmentEnduser2019}.

One common way of implementing \acrshort{eud} for smart home personalization is through web or mobile applications, or even \acrfull{va}~\parencite{barricelliVirtualAssistantsPersonalizing2021,arditoSmartObjectsSmart2018}. These applications enable users to create and customize \textit{routines}, which are sequences of actions that are triggered by certain conditions (e.g., a specific time or date, an event occurrence, a user command). Routines allow users to automate and tailor the behaviour of their smart home control systems according to their preferences and situations. In general, a routine is expressed as a statement composed of a condition --- WHEN something happens --- and one or more actions that are performed if the condition is met --- THEN do something. Usually, in commercial applications, These two components are defined as follows:
\begin{itemize}
    \item Trigger. This is what activates the routine. There are different types of triggers:
          \begin{itemize}
              \item Voice command: The user can say a specific phrase or word to start the routine.
              \item Time: The user can set a specific time and day(s) for the routine to run automatically.
              \item Sunrise/sunset: The user can link the routine to the sunrise or sunset time, or a time interval before or after it, and choose the day(s) to repeat it.
              \item Device: The user can use a sensor device (e.g., smartphone GPS, movement sensor) to trigger the routine when a certain event is detected (e.g., when the user arrives home, the door unlocks and the thermostat adjusts).
          \end{itemize}
    \item Actions. These are the tasks that the routine performs. There are different types of actions:
          \begin{itemize}
              \item Information: provide information such as weather, traffic, etc.
              \item Reminders: remind the user of calendar events, shopping list, etc.
              \item Announcements: send or read messages to the user or others.
              \item Management of connected devices: control smart devices such as lights, plugs, thermostats, door locks, alarm systems, etc.
              \item Media control: play media such as news, music, radio, sleep sounds, etc.
          \end{itemize}
\end{itemize}
