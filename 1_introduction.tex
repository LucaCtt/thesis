\chapter{Introduction}\label{ch:introduction}

The developments in the recent years in the fields of \acrfull{ai} and \acrfull{iot} enabled the emergence of the concept of \acrfull{dt}, which is a powerful tool for modeling and optimizing complex systems in various domains, such as manufacturing, healthcare, and energy saving~\parencite{negriReviewRolesDigital2017,barricelliSurveyDigitalTwin2019,hermannDesignPrinciplesIndustrie2016}. In the context of buildings, \acrshort{dt}s can enable a more efficient and intelligent management of energy, security, and comfort, as well as a better understanding of the interactions between the physical and the digital worlds~\parencite{tagliabueLeveragingDigitalTwin2021,yangDigitalTwinsIntelligent2022}.

This thesis proposes to explore the design and implementation of a \acrshort{dt} for a green smart home, to forecast and optimize the energy consumption of various appliances. This research is significant because it can contribute to the development of a new paradigm for smart and sustainable living, as well as to the advancement of the theoretical and practical knowledge of \acrshort{dt}s in the building sector.

This work is organized as follows: Chapter~\ref{ch:background} provides an overview of the main concepts and technologies that underpin this work. Chapter~\ref{ch:datasets} surveys the existing smart home datasets and their characteristics. Chapter~\ref{ch:hypothetical_home} describes a hypothetical smart home scenario that serves as a case study for the \acrshort{dt}. Chapter~\ref{ch:digital_twin} details the design and implementation of the \acrshort{dt} as a \acrshort{rest} \acrshort{api}. Chapter~\ref{ch:conclusions} summarizes the main contributions of this thesis and outlines some directions for future work.