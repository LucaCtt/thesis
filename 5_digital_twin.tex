\chapter{Digital Twin}\label{ch:digital_twin}

\begin{lstlisting}[language=json,caption={JSON file describing the fridge.},label=fridge_json,float,floatplacement=H]
{
    "id": 4,
    "device": "fridge freezer",
    "manufacturer": "Siemens",
    "model": "",
    "location": "kitchen",
    "modes": [
        {
            "id": 0,
            "name": "off",
            "power_consumption": 0
        },
        {
            "id": 1,
            "name": "on",
            "power_consumption": 208
        }
    ]
}
\end{lstlisting}

\section{Conflict Scenarios}

When simulating the addition of a new routine, a number of conflict scenarios with existing routines may happen. It is assumed that the existing routines are already consistent and conflict-free. The digital twin evaluates the potential conflicts in a sequential order and produces either errors or recommendations, or both. An error occurs when a conflict creates an inconsistency among the routines, e.g. when routines include actions that contradict each other, or that put the power draw over a maximum limit. An error halts the simulation and prevents further evaluation. Recommendations are either warnings about minor conflicts that do not affect the consistency, or suggestions to improve the user experience. A recommendation does not interrupt the simulation and allows for further evaluation. This means that the digital twin can provide multiple recommendations for different conflicts.


\begin{enumerate}[label={\textit{S\arabic*.}}, leftmargin=3.5em]
    \item \textit{An appliance has conflicting operation modes in the same time interval}. This occurs when a simulated routine assigns an appliance a different operation mode than the one assigned by the existing routines in the same time interval. If the simulated routine assigns the same operation mode as the existing routines, there is no conflict. For instance, the AC is set to \textit{heat} from 8:00 to 12:00. A simulated routine that sets the AC to \textit{cold} from 10:00 to 11:00 causes a conflict. The digital twin rejects the simulated routine with an error that specifies the conflicting action and the existing routine and action. It also suggests deactivating the existing routine instead. This may be useful, for example, in weather conditions with sudden temperature changes, where the user may want to adjust the AC temporarily. If the simulated routine does not cause an inconsistency, e.g. it sets the AC to \textit{heat} from 11:00 to 13:00, then the digital twin accepts the simulated routine with a recommendation to modify the existing routine action to set \textit{heat} from 8:00 to 13:00.

    \item \textit{The power consumption exceeds a maximum limit at any time}. The digital twin rejects any routine that causes the power consumption to go beyond the maximum limit at any point. This limit could be the maximum power supply before cut off, e.g. $3kW$ in Italy. For example, if the microwave is in \textit{microwave mode}, the dishwasher is in \textit{intensive} mode, and the simulated routine sets the washing machine to \textit{cotton 60} mode at 14:00, the digital twin produces an error and advises a better start time for the washing machine.

    \item \textit{A better start time can be determined}. The digital twin should find a more economical start time for each action of the simulated routine, using pre-set electricity costs. This is important, for example, if the user has a variable electricity tariff, where it is less expensive to run appliances at night. If the electricity cost is constant throughout the day, this step can be ignored. The digital twin evaluates each action of the simulated routing sequentially, taking into account the best start time of the previous actions. For instance, if the best start time of action 1 of the simulated routine is 15:30, the digital twin should use that as a reference when evaluating action 2.

    \item \textit{An appliance is running or activated manually}. This scenario requires the digital twin to monitor the running state of appliances, in addition to the routines. Before executing any routine actions, the digital twin should check for conflicts with scenarios S1 and S2, using the current state of the appliances as well. If the scenarios produce an error, the conflicting actions should be prevented from running. Actions that are conflict-free can still be executed. Any recommendation produced can be displayed to the user to alert them and offer alternatives on what to do.
\end{enumerate}

\section{Identification of Best Start Times for an Action}

\begin{equation}
    \begin{aligned}
        M &= \text{minutes of the day} = \{m \in \mathbb{R}: 1 \leq m \leq 24*60\}\\
        A &= \text{appliances} = \{a_1, a_2, ..., a_n\}\\
        O_a &= \textit{operation modes of appliance} a \in A = \{o_{a_1}, o_{a_2}, ..., o_{a_n}\}\\
        c_m &= \text{cost of electricity at minute } m \in M\\
        e_m &= \text{total electricity consumption at minute } m \in M\\
        l &= \text{maximum electricity consumption limit}\\
        d_i &= \text{duration of operation mode } i \in O_a\\
        e_s &= \text{electricity consumption of operation mode } i \in O_a\\
        x_m &= \text{binary variable indicating if appliance } a \in A \text{ is running at minute } m \in M\\
        y_i &= \text{binary variable indicating if operation mode } i \in O_a \text{ is set at minute } m \in M\\
        F_m &= \text{total electricity consumption at minute } m \in M\\
        0 &\leq F_m \leq l \quad \forall m \in M\\
        arg\,min_{m \in M}{F_m}
    \end{aligned}
\end{equation}